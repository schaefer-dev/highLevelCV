% exercise sheet with header on every page for math or close subjects
\documentclass[12pt]{article}
\usepackage{german} 
\usepackage[utf8]{inputenc} 
\usepackage{latexsym} 
\usepackage{multicol}
\usepackage{fancyhdr}
\usepackage{amsfonts} 
\usepackage{amsmath}
\usepackage{amssymb}
\usepackage{enumerate}
\usepackage{MnSymbol}
\usepackage[colorlinks=true,urlcolor=blue]{hyperref}
\usepackage{listings}
\usepackage{graphicx}

% Shortcuts for bb, frak and cal letters
\newcommand{\E}{\mathbb{E}}
\newcommand{\V}{\mathbb{V}}
\renewcommand{\P}{\mathbb{P}}
\newcommand{\N}{\mathbb{N}}
\newcommand{\R}{\mathbb{R}}
\newcommand{\C}{\mathbb{C}}
\newcommand{\Z}{\mathbb{Z}}
\newcommand{\Pfrak}{\mathfrak{P}}
\newcommand{\Pfrac}{\mathfrak{P}}
\newcommand{\Bfrac}{\mathfrak{P}}
\newcommand{\Bfrak}{\mathfrak{B}}
\newcommand{\Fcal}{\mathcal{F}}
\newcommand{\Ycal}{\mathcal{Y}}
\newcommand{\Bcal}{\mathcal{B}}
\newcommand{\Acal}{\mathcal{A}}


% Formatierung
\topmargin -2cm 
\textheight 24cm
\textwidth 16.0 cm 
\oddsidemargin -0.1cm

\setlength{\parindent}{0pt}  % !!!!!!! Hier werden leerzeilen erlaubt ohne dass Latex automatisch einrueckt! !!!!!!! %


%Python code Highlighting
\lstset{language=Python, tabsize=3,
        basicstyle=\ttfamily\small, 
        keywordstyle=\color{keywords},
        commentstyle=\color{comments},
        stringstyle=\color{red},
        showstringspaces=false,
        identifierstyle=\color{green}}

% Code-Highlighting Java
%\lstset{language=Java, breaklines=true, showstringspaces=false}
%\begin{lstlisting}
%    	Hier würde der Java-Code hinkommen und entsprechend die Syntax markiert. Selbst einrücken.
%\end{lstlisting}
%ODER:
% \lstinputlisting[language=Java]{name.py}

\begin{document}

% Titel
%\title{\textsc{Hacking}\\ \textsc{Abgabe 0}\\{ \normalsize Gruppe X \hfill Daniel Schäfer (2549458)\\ \hfill Anderer}}
%\maketitle  

% alternativer Titel
\noindent
{\Large \textbf{High-level Computer Vision}} \hfill \textbf{05.05.2016}\\
{\Large \textbf{Exercise 1}} \hfill Thomas Pohl (????)\\
\raggedleft \hfill Guillermo Reyes (2556018)\\
\hfill Daniel Schaefer (2549458)\\
\hfill Dominik Weber (2548553)\\

\pagenumbering{gobble}
\raggedright


\section*{Code Annotations}

\begin{itemize}
    \item 
        Throughout our distance functions we allowed 1D, 2D \textbf{and 3D} histograms. We also implemented support for 3D histograms because we expect this to be a likely case although it didn't happen in our exercises.
\end{itemize}


\section*{Question 1: Image Filtering}

\begin{enumerate}[a)]
    \setcounter{enumi}{2}
    \item 
        % TODO 
        \textbf{What happens when you apply the following filter combinations?}\\
        \begin{enumerate}[1.]
            \item 
                Gaussian Blur
            \item
                filter for the partial y-derivative
            \item
                filter for the partial x-derivative
            \item
                same result as in 3 (commutativity)
            \item 
                same result as in 2 (commutativtiy)
        \end{enumerate}

    \item
        % TODO 
        Comment on the output
\end{enumerate}


\section*{Question 3: Object Identification}

\begin{enumerate}[a)]
    \setcounter{enumi}{2}
    \item 
        % TODO 
        submit the summary of your experiments as part of your solution
\end{enumerate}


\section*{Question 4: Performance Evaluation}

\begin{enumerate}[a)]
    \setcounter{enumi}{2}
    \item 
        % TODO 
        Summary of your observations
\end{enumerate}


\end{document}
