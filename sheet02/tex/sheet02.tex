% exercise sheet with header on every page for math or close subjects
\documentclass[12pt]{article}
\usepackage{german} 
\usepackage[utf8]{inputenc} 
\usepackage{latexsym} 
\usepackage{multicol}
\usepackage{fancyhdr}
\usepackage{amsfonts} 
\usepackage{amsmath}
\usepackage{amssymb}
\usepackage{enumerate}
\usepackage{MnSymbol}
\usepackage[colorlinks=true,urlcolor=blue]{hyperref}
\usepackage{listings}
\usepackage{graphicx}

% Shortcuts for bb, frak and cal letters
\newcommand{\E}{\mathbb{E}}
\newcommand{\V}{\mathbb{V}}
\renewcommand{\P}{\mathbb{P}}
\newcommand{\N}{\mathbb{N}}
\newcommand{\R}{\mathbb{R}}
\newcommand{\C}{\mathbb{C}}
\newcommand{\Z}{\mathbb{Z}}
\newcommand{\Pfrak}{\mathfrak{P}}
\newcommand{\Pfrac}{\mathfrak{P}}
\newcommand{\Bfrac}{\mathfrak{P}}
\newcommand{\Bfrak}{\mathfrak{B}}
\newcommand{\Fcal}{\mathcal{F}}
\newcommand{\Ycal}{\mathcal{Y}}
\newcommand{\Bcal}{\mathcal{B}}
\newcommand{\Acal}{\mathcal{A}}


% Formatierung
\topmargin -2cm 
\textheight 24cm
\textwidth 16.0 cm 
\oddsidemargin -0.1cm

\setlength{\parindent}{0pt}  % !!!!!!! Hier werden leerzeilen erlaubt ohne dass Latex automatisch einrueckt! !!!!!!! %


%Python code Highlighting
\lstset{language=Python, tabsize=3,
        basicstyle=\ttfamily\small, 
        keywordstyle=\color{keywords},
        commentstyle=\color{comments},
        stringstyle=\color{red},
        showstringspaces=false,
        identifierstyle=\color{green}}

% Code-Highlighting Java
%\lstset{language=Java, breaklines=true, showstringspaces=false}
%\begin{lstlisting}
%    	Hier würde der Java-Code hinkommen und entsprechend die Syntax markiert. Selbst einrücken.
%\end{lstlisting}
%ODER:
% \lstinputlisting[language=Java]{name.py}

\begin{document}

% Titel
%\title{\textsc{Hacking}\\ \textsc{Abgabe 0}\\{ \normalsize Gruppe X \hfill Daniel Schäfer (2549458)\\ \hfill Anderer}}
%\maketitle  

% alternativer Titel
\noindent
{\Large \textbf{High-level Computer Vision}} \hfill \textbf{16.05.2016}\\
{\Large \textbf{Exercise 2}} \hfill Thomas Pohl (2537675)\\
\raggedleft \hfill Guillermo Reyes (2556018)\\
\hfill Daniel Schaefer (2549458)\\
\hfill Dominik Weber (2548553)\\

\pagenumbering{gobble}
\raggedright


\section*{Code Annotations}




\section*{Question 3: Region Descriptors}

\begin{enumerate}[a)]
	\setcounter{enumi}{3}
	\item 	
	\textbf{Which descriptor performs better and why?}\\
    %TODO
	
	\item
	\textbf{Summary of observations}\\
    %TODO

\end{enumerate}

\section*{Question 4: Panorama Stitching}
\begin{enumerate}[a)]
%	\setcounter{enumi}{2}
	\item 
	\textbf{Homography estimation with RANSAC.}
     \begin{itemize}
     	\item
     	\textbf{What does homography estimation do?}\\
        According to the lecture Homography is a mapping between two perspective projections with the same center of projection. In the case of \verb!get_ransac_hom.m!, it tries to compute the homography as a transformation matrix that transforms one given set of points to another.  In practise a homography can usually not be computed exactly due to noise in the data. Therefore it has to be estimated, for which it is important to 'ignore' outlier data (possible errors in data collection, \dots) to preserve a realistic and precise estimation.

     	\item
     	\textbf{What are the steps of RANSAC algorithm and why it is useful?}\\
        %TODO

     \end{itemize}
\end{enumerate}


\end{document}
